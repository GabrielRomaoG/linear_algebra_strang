\documentclass{report}

\input{preamble}
\input{macros}
\input{letterfonts}
\usepackage{amsmath}

\title{\Huge{Solutions for Introduction to Linear Algebra 5th -
Gilbert Strang}\\ Chapter1 }
\author{\huge{Gabriel Romão}}
\date{}

\begin{document}

\maketitle
\newpage

\pdfbookmark[section]{ \contentsname }{ toc }
\tableofcontents

\chapter{Problem Set 1.1}

\begin{question}{Describe geometrically (line, plane or all of
    $\mathbb{R}^3$) all linear combinations of }{}
    a.

    $
    \begin{bmatrix}
        2 & 3\\
        4 & 4\\
    \end{bmatrix}
    $
    $
    \begin{bmatrix}
        1\\
        2\\
        3\\
    \end{bmatrix}
    $
    and
    $
    \begin{bmatrix}
        3\\
        6\\
        9\\
    \end{bmatrix}
    $
    b.
    $
    \begin{bmatrix}
        1\\
        0\\
        0\\
    \end{bmatrix}
    $
    and
    $
    \begin{bmatrix}
        0\\
        2\\
        3\\
    \end{bmatrix}
    $
    c.
    $
    \begin{bmatrix}
        2\\
        0\\
        0\\
    \end{bmatrix}
    $
    and
    $
    \begin{bmatrix}
        0\\
        2\\
        2\\
    \end{bmatrix}
    $
    and
    $
    \begin{bmatrix}
        2\\
        2\\
        3\\
    \end{bmatrix}
    $

\end{question}

\sol

\subsection*{ a. }

\[
    \begin{bmatrix}
        1\\
        2\\
        3\\
    \end{bmatrix}
    * 3 =
    \begin{bmatrix}
        3\\
        6\\
        9\\
    \end{bmatrix}
\]
, line

\begin{question}{5}{}

\end{question}

\sol

\begin{gather*}
    \mathbf{u} =
    \begin{bmatrix}
        1\\
        2\\
        3\\
    \end{bmatrix},
    \mathbf{v} =
    \begin{bmatrix}
        -3\\
        1\\
        -2\\
    \end{bmatrix},
    \mathbf{w} =
    \begin{bmatrix}
        2\\
        -3\\
        -1\\
    \end{bmatrix} \\
    \mathbf{u} + \mathbf{v} + \mathbf{w} =
    \begin{bmatrix}
        0\\
        0\\
        0\\
    \end{bmatrix} \\
    2\mathbf{u} + 2\mathbf{v} + \mathbf{w} =
    \begin{bmatrix}
        -2\\
        3\\
        1\\
    \end{bmatrix}
\end{gather*}

c and d:

\begin{gather*}
    \begin{bmatrix}
        2\\
        -3\\
        -1\\
    \end{bmatrix} =
    c
    \begin{bmatrix}
        1\\
        2\\
        3\\
    \end{bmatrix} +
    d
    \begin{bmatrix}
        -3\\
        1\\
        -2\\
    \end{bmatrix} \\
    c = -1, d = -1
\end{gather*}

\newpage

\chapter{Problem Set 1.2}

\begin{question}{1}{}
    \begin{gather*}
        \mathbf{u} \cdot \mathbf{v} =
        \begin{bmatrix}
            -2.4\\
            2.4\\
        \end{bmatrix}
    \end{gather*}
    \begin{gather*}
        \mathbf{u} \cdot ( \mathbf{v} + \mathbf{w} ) = \\
        =
        \begin{bmatrix}
            -0.6\\
            0.8\\
        \end{bmatrix} \cdot
        \begin{bmatrix}
            5\\
            5\\
        \end{bmatrix} \\
        =
        \begin{bmatrix}
            -3\\
            4\\
        \end{bmatrix}
    \end{gather*}
\end{question}

\begin{question}{2}{}
    \begin{gather*}
        \lVert \mathbf{u} \rVert = \sqrt{\mathbf{u} \cdot \mathbf{u}} \\
        = \sqrt{(-0.6 * -0.6) + (0.8 * 0.8)} \\
        = \sqrt{0.36 + 0.64} \\
        = 1
    \end{gather*}
    \begin{gather*}
        \lVert \mathbf{v} \rVert = \sqrt{(4*4) + (3 * 3)} = 5
    \end{gather*}
\end{question}

\begin{question}{4}{}
    if the lenght of $\mathbf{v}$ is 1, $\mathbf{v}$ is a unit vector.\\
    a.
    \begin{gather*}
        \vlen{\mathbf{v}} = 1 \\
        \mathbf{v} \cdot \mathbf{v} = 1 \\
        \leadsto \mathbf{v} \cdot \mathbf{-v} = -1
    \end{gather*}
    b.
    \begin{gather*}
        \begin{bmatrix}
            v_0 + w_0\\
            \vdots \\
            v_n + w_n\\
        \end{bmatrix} \cdot
        \begin{bmatrix}
            v_0 - w_0\\
            \vdots \\
            v_n - w_n\\
        \end{bmatrix} = \\
        = \sum_{i = 0}^{n} { v_0^2 - w_0^2 } \\
        = \sum_{i = 0}^{n} { v_0^2 } - \sum_{i = 0}^{n} {
        w_0^2 } \\
        = \vlen{\mathbf{v}} - \vlen{\mathbf{w}} = 0
    \end{gather*}
\end{question}

\begin{question}{6}{}
    a.
    \begin{gather*}
        \mathbf{w} \cdot \mathbf{v} = 0 \\
        (w_1 * 2) + (w_2 * -1) = 0 \\
        2w_1 - w_2 = 0 \\
        2w_1 = w_2
    \end{gather*}
\end{question}

\begin{question}{7}{}
    \begin{gather*}
        \cos \theta = \displaystyle\frac{\mathbf{w} \cdot \mathbf{v}}{\vlen{\mathbf{w}} \cdot \vlen{\mathbf{v}}}
    \end{gather*}
    c.
    \begin{gather*}
        \mathbf{w} \cdot \mathbf{v} = \frac{1}{\sqrt{3}} \cdot \frac{-1}{\sqrt{3}} \\
        = -1 + 3 = 2 \\
        \vlen{\mathbf{w}} = \sqrt{(-1 * -1) + (\sqrt{3} * \sqrt{3}) } \\
        = \sqrt{1 + 3 } = 2 \\
        \vlen{\mathbf{v}} = \sqrt{(1 * 1) + (\sqrt{3} * \sqrt{3}) } \\
        = \sqrt{1 + 3 } = 2
    \end{gather*}

    \begin{gather*}
        \cos \theta = \frac{2}{2*2} = \frac{1}{2} \\
        \theta = \frac{\pi}{3}
    \end{gather*}
\end{question}

\begin{question}{12}{}
    \begin{gather*}
        (\mathbf{w} - c \mathbf{v}) \cdot \mathbf{v} = 0 \\
        (
        \begin{bmatrix}
        1\\
        5\\
        \end{bmatrix}
        - c 
        \begin{bmatrix}
        1\\
        1\\
        \end{bmatrix}) 
        \cdot
        \begin{bmatrix}
        1\\
        1\\
        \end{bmatrix}
        = 0 \\
        \begin{bmatrix}
        1 - c\\
        5 - c\\
        \end{bmatrix}
        \cdot
        \begin{bmatrix}
        1\\
        1\\
        \end{bmatrix}
        = 0 \\
        (1 - c) + (5 - c) = 0 \\
        6 - 2c = 0 \\
        c = 3
    \end{gather*}
    \begin{gather*}
        (\mathbf{w} - c \mathbf{v}) \cdot \mathbf{v} = 0 \\
        \mathbf{w} \cdot \mathbf{v} - c \mathbf{v} \cdot \mathbf{v} = 0 \\
         - c \mathbf{v} \cdot \mathbf{v} = \mathbf{w} \cdot \mathbf{v}   \\
         c = - \frac{\mathbf{w} \cdot \mathbf{v}}{\mathbf{v} \cdot \mathbf{v}}   \\
    \end{gather*}
\end{question}

\begin{question}{13}{}
    \begin{gather*}
        \mathbf{w} \cdot (1, 0, 1) = 0 \\
        w_1 + w_2 = 0 \\
        w_1 = -w_2 \\
        \mathbf{w} = (1, 0, -1)
    \end{gather*}
    \begin{gather}
        \mathbf{v} \cdot \mathbf{w} = 0 \\
        \mathbf{v} \cdot (1, 0, 1)= 0 
    \end{gather}
    2.1
    \begin{gather*}
        (v_1, v_2, v_3) \cdot (1, 0, -1) = 0 \\
        v_1 - v_3 = 0 \\
        v_1 = v_3
    \end{gather*}
    2.2
    \begin{gather*}
        (v_1, v_2, v_3) \cdot (1, 0, 1) = 0 \\
        v_1 + v_3 = 0 \\
        v_1 = -v_3 \\
        v_1 = 0 \\ 
        v_3 = 0
    \end{gather*}
    \begin{gather*}
        \mathbf{v} = (0, v_2, 0), v_2 \in \mathbb{R}
    \end{gather*}
\end{question}
\begin{question}{19}{}
    \begin{gather*}
        \vlen{\mathbf{v} + \mathbf{w}} ^ 2 \\
    \vlen{\mathbf{u}} \\
    \mathbf{u} \cdot \mathbf{u} \\
    \mathbf{u} \cdot (\mathbf{v} + \mathbf{w}) \\
    \mathbf{u} \cdot \mathbf{v} + \mathbf{u} \cdot \mathbf{w} \\
    ( \mathbf{v} \cdot \mathbf{v} + \mathbf{v} \cdot \mathbf{w}) + (\mathbf{w} \cdot \mathbf{w} + \mathbf{v} \cdot \mathbf{w}) \\
    \leadsto \mathbf{v} \cdot \mathbf{v} + 2 \mathbf{v} \cdot \mathbf{w} + \mathbf{w} \cdot \mathbf{w}
    \end{gather*}
\end{question}

\end{document}
